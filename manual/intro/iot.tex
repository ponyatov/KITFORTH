\clearpage
\secly{\iot: универсальный протокол для Интернета вещей}

С развитием и внедрением технологий Интернета вещей начинают играть другие грани
\F а: легкая портируемость на любые аппаратные платформы с сохранением
низкоуровневого доступа к ``железу'', и крайне низкие требования к ресурсам.
16-битная \F-система вполне способна работать всего на нескольких Кб ОЗУ, и
имеет очень компактный код (имеет значение если вы хотите обновлять IoT
устройства по воздуху).

\emph{Для IoT приложений \F\ интересен} с точки зрения его использования
\emph{как универсальный \textbf{перепрограммируемый} протокол, который может
быть использован} не только узлами сети для обмена данными и командами, но и
\emph{непосредственно человеком}, через самый универсальный тип
интерфейса\ --- последовательный канал связи, и командную строку.

\clearpage
В IoT существует значимая проблема\ --- отсутствие универсального и одновременно
очень компактного протокола верхнего уровня для представления данных и передачи
команд, который бы был удобен для (де)сери\-али\-зации на устройствах с очень
простыми микроконтроллерами, но при этом максимально универсальным. MQTT
обеспечивает транспорт данных, но никак не описывает форматы и семантику.

\kf\ --- экспериментальный проект по применению \F a в \iot\ как
универсального языка и формата обмена данными. Предполагается, что первоначально
используемая непосредственно пользователем консоль будет постепенно заменяться
\begin{itemize}[nosep]
  \item 
графическими приложениями (на мобильных устройствах), 
  \item 
скриптами автоматизации на домашних роутерах, одновременно используемых как
контроллер умного дома
  \item 
десктопными приложениями на ПК,
  \item
и устройства будут напрямую обмениваться командами друг с другом
\end{itemize} 

\clearpage
Так как
\begin{itemize} 
  \item 
используются базовые интерфейсы передачи данных:\\UART, последовательные
каналы по WiFi/BT, telnet/SSH\\(возможна пакетная передача через SMS, LoraWAN,
NB-IoT)
  \item 
доступные на всех типах компьютеров, и 
  \item 
доступные напрямую в ОС без каких-либо дополнительных (закрытых) библиотек,
  \item 
полностью открытый и специфицированный текстовый протокол для команд\ --- \F-код 
\end{itemize}
предполагается что любой желающий сможет легко писать приложения для своих нужд,
и интегрироваться с любыми существующими системами.

