\clearpage
\secly{Программирование для кофеварки ?!}\label{intro}

\noindent
Бадумтс-с-с! Вся семья сбежалась смотреть на чудо\ --- \textit{руководство
программиста}, выпавшее из коробки\note{пока доступно только в электронном виде
на сайте проекта:\\\url{https://github.com/ponyatov/KITFORTH/releases/laetst/} в
виде \file{.pdf}\ файла} с вафельницей или минипечкой. К сожалению, не с
чайником или плитой: технологию IoT\note{интернет вещей, Internet-of-Things}\
поддерживают пока только два устройства, которые я кое-как смог запихать на
кухню:
\begin{itemize}
  \item 
Микропечь
\hremd{Kitfort KT-1709}{https://kitfort.ru/catalog/duhovki/17341/}
  \item 
Вафельница для бельгийских вафель
\hremd{Kitfort KT-1646}{https://kitfort.ru/catalog/vafelnitsa/20584/}
\end{itemize}

\noindent
Хотя вряд ли, скорее всего вы купили комплект ``микродуховка для программиста''
сами себе, потому что вас зацепила идея пощщупать технологию интернета вещей
вживую, или просто получили в подарок.

\clearpage
\secly{Откуда взялась идея}

Технология IoT была для вас только хайпом, потому что до сих пор
цена на ``умные'' бытовые устройства зашкаливала, чтобы взять их просто ``на
попробовать''. Другая проблема, куда более важная\ --- каждый производитель
пилит свой собственный набор протоколов, и полностью закрытое программное
обеспечение. Особенно весело с этим у китайской продукции с AliExpress, весело
продающейся и в ближайших магазинах под нескольми дешевыми типа брендами.
Вы покупаете устройство, например action камеру. В комплекте идет приложение под
Android (то есть все остальные виды компьютеров сразу в пролете). Допустим вы
его ставите, дав разрешение на все возможные действия, включая международные
звонки и снятие отпечатков пальцев. Через два месяца приложение обновляется на
Play Store, и выясняется что теперь оно требует как минимум Android 9, и не
работает ни на одном из ваших устройств.
