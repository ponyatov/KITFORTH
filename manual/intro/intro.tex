\secly{Программирование для кофеварки ?!}\label{intro}

\noindent
Бадумтс-с-с! Вся семья сбежалась смотреть на чудо\ --- \textit{руководство
программиста}, выпавшее из коробки\note{пока доступно только в электронном виде
на сайте проекта:\\\url{https://github.com/ponyatov/KITFORTH/releases/laetst/} в
виде \file{.pdf}\ файла} с вафельницей или минипечкой. К сожалению, не с
чайником или плитой: технологию IoT\note{интернет вещей, Internet-Of-Things}\
поддерживают только два устройства, которые я кое-как смог запихать на кухню:
\begin{itemize}[nosep]
  \item 
Микропечь
\hremd{Kitfort KT-1709}{https://kitfort.ru/catalog/duhovki/17341/}
  \item 
Вафельница для бельгийских вафель
\hremd{Kitfort KT-1646}{https://kitfort.ru/catalog/vafelnitsa/20584/}
\end{itemize}

\noindent
Хотя вряд ли, скорее всего вы купили комплект ``микродуховка для программиста''
сами себе, потому что вас зацепила идея пощщупать технологию интернета вещей
вживую, или просто получили в подарок.
