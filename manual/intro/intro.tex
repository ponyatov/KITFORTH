\clearpage
\secly{Программирование для кофеварки ?!}\label{intro}

\noindent
Бадумтс-с-с! Вся семья сбежалась смотреть на чудо\ --- \textit{руководство
программиста}, выпавшее из коробки\note{пока доступно только в электронном виде
на сайте проекта:\\\url{https://github.com/ponyatov/KITFORTH/releases/latest/} в
виде \file{.pdf}\ файла} с вафельницей или минипечкой. К сожалению, не с
чайником или плитой: технологию IoT\note{интернет вещей, Internet-of-Things}\
поддерживают пока только два устройства, которые я кое-как смог запихать на
кухню:
\begin{itemize}
  \item 
Микропечь
\hremd{Kitfort KT-1709}{https://kitfort.ru/catalog/duhovki/17341/}
  \item 
Вафельница для бельгийских вафель
\hremd{Kitfort KT-1646}{https://kitfort.ru/catalog/vafelnitsa/20584/}
\end{itemize}

\noindent
Хотя вряд ли, скорее всего вы купили комплект ``микродуховка для программиста''
сами себе, потому что вас зацепила идея пощщупать технологию интернета вещей
вживую, пользуясь готовым устройством из коробки.

\clearpage
\secly{Откуда взялась идея}

Технология IoT была для вас только хайпом, потому что до сих пор цена на
``умные'' бытовые устройства зашкаливала, чтобы взять их просто ``на
попробовать''. Другая проблема, куда более важная\ --- каждый производитель
пилит свой собственный набор протоколов, и полностью закрытое программное
обеспечение. Особенно весело с этим у китайской продукции с AliExpress, весело
продающейся и в ближайших магазинах под нескольми дешевыми типа брендами.
Вы покупаете устройство, например action камеру. В комплекте идет приложение под
Android\note{то есть все остальные виды компьютеров сразу в пролете}.
Допустим вы его ставите, дав разрешение на все возможные действия, включая
международные звонки и снятие отпечатков пальцев. Через два месяца приложение
обновляется на Play Store, и выясняется что теперь оно требует как минимум
Android 99, и не работает ни на одном из ваших телефонов.

\begin{framed}
\noindent
Закрытая прошивка имеет одно очень важное свойство\ --- вы можете использовать
устройство только так, как это заложил производитель. Все возможные проблемы и
ошибки в прошивке, приводящие к неправильному функционированию,
\emph{принципиально неизлечимы}. Даже если предусмотрена возможность ее
обновления, чаще всего производитель не исправляет некричные ошибки, забивает на
поддержку сразу после выхода следующей модели, и уж тем более не предоставляет
никакой технической поддержки пользователям, хотящим странного.
\end{framed}

\noindent
Для дорогих брендовых моделей ситуация чуть получше, выходит 1-2 обновления,
поддерка старых моделей прекращается немного позже, а опции и нестандартные
решения предоставляются по стоимости, сравнимой с новым устройством следующего
поколения.

\clearpage
\secly{OpenSource \& OpenHardware}

Идеология OpenSource предлагает совершенно другие возможности даже неопытному
пользователю, не говоря о тех кто в теме \term{embedded разработки}.

Производитель оборудования полностью снимает с себя ответственность по
поддержке продукта, предоставляя только апаратную базу, и прошивку,
обеспечивающие вам только базовые функции, доступные из коробки. Взамен
производитель предоставляет полный исходный код прошивки, и общую схему
устройства, на которой указано куда и что подключено.

В результате вы как конечный пользователь, от этого только выигрываете, так как
\emph{поддержку берет на себя сообщество пользователей}. Общаясь в сети на
открытых ресурсах\note{как минимум веб-форум производителя}, вы можете найти
способ реализовать любые ваши хотелки, даже если совершенно не разбираетесь в
программировании\ --- если идея интересна, найдутся достаточно опытные
пользователи, способные ее реализовать, и поделиться с вами результатом.

Правильно написанная документация, и помощь открытого сообщества поможет вам при
желании освоить программирование устройства на самом минимальном уровне, и при
этом делать достаточно сложные и нестандартные вещи. \emph{Даже просто повторяя
и комбинируя шаги, описанные в постах на веб-форуме, вы можете получить
возможности, недостижимые в традиционных даже самых дорогих моделях}. Для тех
же, кто уже имеет некоторые навыки, особенно в электронике, открываются
совершенно неограниченные возможности.

Если производитель закроет производство вашей модели, даже через несколько лет
вы сможете так же пользоваться вашей умной плитой, добавлять новые фишки и
нестандартные применения, до тех пор пока устройство в принципе способно
работать.

